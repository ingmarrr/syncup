\documentclass[12pt]{article}

\usepackage{tabularx}

\begin{document}

\title{\textsc{Cross-Platform Mobile Development \linebreak  \large{Assignment 2} }}
\author{\textsc{Sandu Crucerescu - idiot}}
\date{\textsc{\today}}

\maketitle
\pagebreak

\tableofcontents
\pagebreak

\section{Introduction}

Our concept revolves around a family/couple sharing application named "Syncup." 
This innovative app will facilitate users in sharing various essential aspects of their lives,
 including shopping lists, calendars, to-do lists, and expenses, among other features, which we'll delve into shortly.

"Syncup" derives its name from its core purpose: to synchronize daily activities and planning 
among users. While some of the features offered by our app may already exist individually 
on Google and Apple platforms, a common challenge lies in achieving seamless interoperability 
between these applications. Furthermore, these existing apps tend to focus on isolated functions, 
such as Google Calendar, whereas "Syncup" will consolidate all these individual features into a single,
 cross-platform application. This approach creates synergies between different aspects of users' lives, 
 allowing, for instance, the integration of shopping lists with expense tracking.

\section{Target User}

User Persona 1: Family Organizer - Sarah

Demographics:
Age: 35
Gender: Female
Family Status: Married with two young children (ages 6 and 8).
Needs/Expectations:
Sarah is looking for an app that simplifies her family's hectic schedule. She wants an efficient way to share calendars and to-do lists with her spouse.
She desires a shared shopping list to streamline grocery shopping and meal planning.
Sarah is conscious of her family's budget, so expense tracking and financial management features are important.
She values user-friendly interfaces and expects the app to be intuitive and easy to navigate.
\\
\\
User Persona 2: Young Couple - Alex and Emily
Demographics:
Age: Alex (28), Emily (26)
Gender: Both Male and Female
Relationship Status: Newlywed couple.
Needs/Expectations:
Alex and Emily want an app to help them coordinate their lives as a newly married couple.
They are interested in a shared calendar to keep track of personal and joint events.
The ability to collaboratively manage to-do lists for household tasks is essential.
They seek a streamlined way to split expenses and track their spending.
User privacy is crucial, as they want to share some information but keep other aspects of their lives private from each other.
\\
\\
User Journey 1: Sarah - The Family Organizer
Discovery and Onboarding:
Sarah discovers Syncup through a friend's recommendation and downloads the app from the app store.
She creates her account and sets up her family profile, adding her spouse and children.
Daily Usage:
Sarah uses the shared calendar to schedule her children's activities, work meetings, and family events.
She adds items to the shared shopping list throughout the day.
In the evening, she reviews and manages family expenses and budgets using the app's financial tools.
Positive Outcomes:
Syncup has helped Sarah and her family stay organized, reduce scheduling conflicts, and better manage their expenses.
Sarah appreciates the convenience of having all these features in one app.
\\
\\
User Journey 2: Alex and Emily - The Young Couple
Registration and Customization:
Alex and Emily sign up for Syncup after their wedding and customize their profiles.
They set up a shared calendar and a to-do list for household chores.
Day-to-Day Use:
They use the shared calendar to coordinate date nights, work schedules, and social events.
The couple collaboratively manage their household to-do list, checking off tasks as they complete them.
Financial Planning:
Alex and Emily use the expense tracking feature to log shared expenses like rent, utilities, and groceries.
They appreciate the app's ability to calculate how much each of them owes for shared expenses.
Privacy Considerations:
Emily occasionally uses private notes within the app to plan surprises for Alex without him seeing.
Positive Outcomes:
Alex and Emily feel more organized and connected as a couple, thanks to Syncup's features.
They've improved their financial transparency and communication.
In addition to these user personas and journeys, Syncup may also cater to specific sub-groups, such as elderly users seeking simplified interfaces, or single parents looking for co-parenting tools. These sub-groups may have unique characteristics and requirements that the app can accommodate through tailored features or settings.

\pagebreak
\section{Product Niche}
Our app is designed to stand out in a pool of many management apps. There two major platforms Google and Apple, 
both have management apps like GApps and Apples own applications which can be managed thought a family plan. The down side of these
platforms is for one apple apps do not work on google, and in order for user to use the GApps they have to download around a dozen or
so apps which in return multiplies the number of applications with the same functionality on their devices. Out app aims to 
reduce clutter of applications on ones device and have the same amount of functionality. Our application will also implement missing functionality
from other application to make it more intuitive and user friendly. In addition the application will be able
to connect to a bank account(for example: a shared one) and display statistic on monthly payments, in addition to this different budgets 
can also be created which will notify the users if the threshold is about to pass. 

\pagebreak
\section{Features \& Functionality}

\subsection{Features}

We will have different actors in our application, each with different features and functionality. The actors are as follows:

\begin{itemize}
    \setlength\itemsep{0em}
    \item Group Member
    \item Group Admin
\end{itemize}

Depending on their level of authorization, users will have different rights and access to the application's features. 

\setlength{\tabcolsep}{3pt}

\begin{table}[h]
    \small
    \begin{tabularx}{\textwidth}{|p{12pt}|p{85pt}|X|X|}
        \hline  & User & Requirement & Goal \\
        \hline 1 & User & Registration & Create a profile, and login, reset password \\ 
        \hline 2 & Group Member & View Calendar & View plans and Events \\
        \hline 3 & Group Member & Edit Calender & Add/Remove/Edit plants/activies \\
        \hline 4 & Group Member & View Shopping List & View items to be bought \\
        \hline 5 & Group Member & Edit Shopping List & Add/Remove/Edit items to be bought \\
        \hline 6 & Group Member & View To-Do List & View tasks to be done \\
        \hline 7 & Group Member & Edit To-Do List & Add/Remove/Edit tasks to be done \\
        \hline 8 & Group Admin & Add/Remove Group Member & Add/Remove Group Member \\
        \hline 9 & Group Admin & Add/Remove Group Admin & Add/Remove Group Admin \\
        \hline 10 & Group Admin & Manage Permissions & Manage Permissions \\
        \hline 11 & Group Admin & Assign Budgets & Assign Budgets/Allowance \\
        \hline 12 & User & Receive notifications & Receive notifications about upcoming events, or reminders \\
        \hline
    \end{tabularx}
\end{table}

\subsection{Requirements}

\subsubsection{Technical details}

Android or IOS device. Screen size of 6.1 inches or larger, at least 200mb of storage.
Android 10 and IOS 13.0 or later. The app wil interact with payment gateways, and outside servers. 


\section{Development Plan}

\begin{table}[h]
    \begin{tabularx}{\textwidth}{|X|p{100pt}|}
        \hline  Feature & Estimated Days \\
        \hline  Basic setup(database, models, etc.) & 7 \\
        \hline  Calendar Functionality & 5-7 \\
        \hline  Shopping List Features & 5-7 \\
        \hline  To Do list Features & 3-4 \\
        \hline  Admin Functionality & 4-5 \\
        \hline  Finance Functionality & 10 \\
        \hline
    \end{tabularx}
\end{table}

\subsection{Risks during implementation}
\begin{itemize}
    \setlength\itemsep{-1em}
    \item   The implementation of payment gateways might prove a challenge to implement. \\
    \item   Implementation might face technical hurdles in integrating multiple features seamlessly. \\ 
    \item   Ensuring the app works smoothly on both IOS and Android might prove challenging. \\ 
    \item   Attracting and encouraging users to adopt out app. \\
    \item   Handling sensitive user data such as financial information and personal calendars could pose security risks. \\
\end{itemize}


\end{document}
